\documentclass{beamer}

% Perussäädöt
\usepackage[T1]{fontenc}
\usepackage[utf8]{inputenc}
\usepackage[finnish]{babel}

\usepackage{url}

% Fontti: Linux Libertine
\usepackage{libertine}

\usetheme{default}

% Otsikkosivun tiedot
\title{Kaiken yksityisen kopioinnin ja jakamisen laillistaminen?}
\author{
  Jaakko Luttinen
}
\date{4. joulukuuta 2012}

\begin{document}

% Lado otsikkosivu
\maketitle

% Sisällysluettelo
\begin{frame}{Sisältö}
  \tableofcontents
\end{frame}


\section{Ehdotus}

\begin{frame}{Ehdotus}

  \Large Kaikki yksityiseen käyttöön tapahtuva epäkaupallinen
  kopiointi ja jakaminen on laillistettava.
  
\end{frame}

\section{Perustelut}

\begin{frame}{Perustelut}
  \Large
  Ei taloudellista haittaa tekijöille
\end{frame}

\begin{frame}{Perustelut}
  \Large
  Yhteiskunta hyötyy
\end{frame}

\begin{frame}{Perustelut}
  \Large
  Kuluttajien yksityisyys säilyy
\end{frame}

\begin{frame}{Perustelut}
  \Large
  Tekijät saavat näkyvyyttä
\end{frame}

\section{Kritiikki}

\begin{frame}{Kritiikki}
  \Large
  Liiketoimintamalleja huonosti
\end{frame}

\begin{frame}{Kritiikki}
  \Large 
  Epäkaupallisuuden epämääräisyys
\end{frame}

\begin{frame}{Kritiikki}
  \Large
  Teoksista tulee suljettuja
\end{frame}

\begin{frame}{Kritiikki}
  \Large
  Valvonnan vaikeudesta ei seuraa laillisuus
\end{frame}

\section{Muita keinoja}

\begin{frame}{Muita keinoja}
  \Large
  \begin{itemize}
  \item Tekijänoikeuden suoja-ajan lyhentäminen
  \item Kopiointisuojausten kieltäminen
  \item Hyvitysmaksun poistaminen
  \item Avoimen sisällön tuottaminen
  \end{itemize}
\end{frame}

\end{document}
%%% Local Variables: 
%%% mode: latex
%%% TeX-master: t
%%% End: 
