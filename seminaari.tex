\documentclass[titlepage,12pt]{article}

% Perussäädöt
\usepackage[T1]{fontenc}
\usepackage[utf8]{inputenc}
\usepackage[finnish]{babel}

% Fontti: Linux Libertine
\usepackage{libertine}

% Viitteiden hallinta
\usepackage[authoryear,round]{natbib}
\usepackage[nottoc]{tocbibind}

% Otsikkosivun modaus
\usepackage{titling}
\preauthor{
  \vfill \large
  \begin{flushright}
    }

    \postauthor{
  \end{flushright}
}

% Abbreviations
%\usepackage[⟨options ⟩]{nomencl}
%\makenomenclature

% Riviväli 1.5
\usepackage{setspace}
\onehalfspacing

% Otsikkosivun tiedot
\title{Tekijänoikeudet}
\author{
  Seminaarityö\\
  \hfill Syksy 2012\\
  Jaakko Luttinen\\
  $62838$F\\
  Opponentti: N.N.
}
\date{ }



\begin{document}

% Viitteiden asetuksia. Babelin takia nämä pitää olla \begin{document}
% jälkeen.
\renewcommand\refname{Lähteet}
\renewcommand\bibname{Lähteet}
\bibliographystyle{plainnat}

% Lado otsikkosivu
\maketitle

% Sisällysluettelo
\tableofcontents
\pagebreak


\section{Johdanto}


Yksityisen kopioinnin lyhyt historia.

Mitä yksityinen kopiointi tällä hetkellä on?

Mitä piraattipuolue haluaa yksityiseksi kopioinniksi?

Millä perusteilla?

Mitä vikaa näissä perusteissa on?

Olisiko muita ratkaisuja?

\section{Piraattipuolueen ehdotuksen perustelut}

\subsection{Ei taloudellista haittaa tekijöille}

Yksityisessä kopioinnissa ei ole kyse varastamisesta, koska
kopioidessa teos ei poistu alkuperäiseltä omistajalta.  Tekijät
väittävätkin, että haitta ilmeneekin vähenevänä myyntinä, koska
kuluttajat voivat hankkia teoksen ilmaiseksi kopioimalla.
Piraattipuolueen mukaan tämä ei kuitenkaan pidä paikkaansa.

Ensinnäkin kulutustottumukset ovat ylipäänsä muuttuneet (musiikin
käyttö vähentynyt, internetin selailu lisääntynyt) ja ilmaistuotanto
tarjoaa laillisen ilmaisen vaihtoehdon maksulliselle
sisällölle.\footnote{Ks. wat} Toisekseen kuluttajat saattavat ostaa
teoksen joka tapauksessa, jos pitävät siitä, ja ilmaiseksi
hyödyntävien aiheuttamat tappiot korvautuvat niillä uusilla ostajilla,
jotka eivät ilman ilmaista versiota olisi kokeilleet ja päätyneet
ostamaan teosta.\footnote{Ks. wat}

Laittoman kopioinnin vaikutuksia myyntilukuihin on tutkittu
runsaasti.\footnote{83-85} Arviot myynnin vähenemisestä vaihtelevat
muutamasta prosentista jopa 30 prosenttiin.  Mutta kuten todettiin
yllä, myynnin vähentymiseen voi vaikuttaa muitakin täysin laillisia
tekijöitä.  Kun on tutkittu suoraan tiedostonjakoverkkojen käytön
vaikutusta myyntiin, ei ole havaittu mitään
vaikutusta.\footnote{83-85}


\subsection{Yhteiskunta hyötyy}

%Nolla kustannus, positiivinen hyöty

%Jos kaikille voidaan antaa karkki ilmaiseksi, niin eikö se kannata?

Teokset ovat kulttuurista pääomaa, jotka lisäävät ihmisten
hyvinvointia.  Jokainen ihminen siis saa hyötyä teoksista joihin
pääsee käsiksi ja täten myös koko yhteiskunta hyötyy yksittäisten
ihmisten joukkona.  Jos nämä teokset ovat digitaalisessa muodossa,
niitä voidaan jakaa käytännössä kaikille ihmisille olemattomalla
kustannuksella.  Kun kerran on mahdollista lisätä koko kansan
hyvinvointia ilman kustannuksia, eikö se olisi yhteiskunnalle järkevä
tapa toimia?  Jos yksi leipä voitaisiin ilman lisäkustannuksia
kopioida kaikkien ruuaksi, eikö se olisi kannattavaa?

``Varastamisessa on kyse siitä, että jokin esine, jolla on arvoa,
vaihtaa omistajaa.  Yhteiskunnan elintaso ei kohoa, mutta varkauden
uhrin elintaso laskee. Kopioinnissa asia, jolla on arvoa, monistuu.
Yhteiskunnan elintaso kohoaa, koska yhä useammilla on yhä enemmän
arvokkaita asioita hallussaan.  Kopioinnissa ei ole uhria, jonka
elintaso laskisi.''\footnote{Ks. \cite{wat?}}

\subsection{Kuluttajien yksityisyys säilyy}

Laittoman yksityisen kopioinnin valvonta uhkaa viestintäsalaisuutta ja
oikeusturvaa.  Koska kopiointi tapahtuu muun verkkoliikenteen
joukossa, tehokas valvonta vaatisi kaiken verkkoliikenteen
seuraamista, jolloin yksityisyys olisi uhattuna.  Valvonnan
tehostaminen todennäköisimmin vain johtaisi tiukempaan
verkkoliikenteen salaamiseen ja anonymisointiin.\footnote{cite}

Toisaalta toimintaa voidaan koittaa ehkäistä sensuroimalla laittomassa
kopioinnissa hyödynnettäviä sivustoja.\footnote{PirateBay Elisa}
Tällaiset sensuurit ovat kuitenkin sekä tehottomia että sananvapautta
loukkaavia.\footnote{http://piraattipuolue.fi/2011/10/the-pirate-bay-sensuuri-uhka-sananvapaudelle
  ja
  http://piraattipuolue.fi/2011/10/piraattipuolue-tuomitsee-raesaesen-puheet-nettisensuurin-laajentamisesta}


\subsection{Tekijöille näkyvyyttä}

Digitaalisten teosten leviäminen ilmaiseksi kopioimalla voidaan nähdä
myös tekijöiden kannalta hyödyllisenä.  Kuluttajat voivat kokeilla
teosta ja ostaa sen, mikäli pitävät.  Ilman kokeilumahdollisuutta
kuluttaja ei olisi välttämättä päätynyt tekemään ostosta.  Tällä
tavalla erityisesti vähemmän tunnetut tekijät saavat markkinoitua
itseään tehokkaasti.  Vaikka kuluttaja ei ostaisikaan ilmaiseksi
lataamaansa tuotetta, kopiointi voi silti koitua tekijän hyödyksi,
mikäli kopioija päätyykin esimerkiksi ostamaan fanituotteita tai lipun
keikalle.


\section{Piraattipuolueen ehdotuksen kritiikki}

\subsection{Riittämättömät liiketoimintamallit}

Yksi merkittävä uhka Piraattipuolueen määrittämän yksityisen
kopioinnin sallimisessa on luovan työn muuttuminen taloudellisesti
kannattamattomaksi.  Yhteiskunnan (ja yksilöiden) kannalta on parasta,
kun mahdollisimman suurta hyötyä voidaan jakaa mahdollisimman monelle.
Yksityinen kopiointi mahdollistaa mahdollisimman monelle jakamisen,
mutta mikäli sen seurauksena teosten laatu ja määrä laskisi (eli hyöty
pienenisi), voi yhteiskunnan näkökulmasta suurin kokonaishyöty olla
saavutettavissa, kun mahdollistetaan tekijöille kannattavia
liiketoimintamalleja laadukkaiden teoksien tuottamiseksi.  Näihin
kysymyksiin Piraattipuolue ottaa kantaa kovin ylimalkaisesti.

% Mikä malli ikinä olisikaan, se on siinä mielessä tasapuolinen, että
% jokaisella ihmisellä kuitenkin on aivan yhtäläinen mahdollisuus
% valita, ryhtyykö luovalle alalle.

% Lisäksi paljon hoettu toteamus ``tekijällä on oikeus saada korvaus''
% ei pidä paikkaansa, vaan paremminkin pitäisi sanoa, että "tekijällä
% on oikeus pyytää korvausta".

Yksi ehdotettu rahoitusmuoto olisi ihmisten vapaaehtoisuus.
Kuluttajat lataisivat teokset ilmaiseksi, mutta mikäli tykkäävät
niistä, maksavat tekijälle osoittaakseen tukea.

jokapiraatinoikeus s.90, ihmiset maksaisivat koska se on hyvä tapa..


Palvelut tai fyysiset tuotteet varsinaisena myytävänä

Rahoittajat ja tilaajat

Lahjoitukset ja fanituotteet


kokeilumalli toimii vain jos teosta hyödynnetään useita kertoja
(elokuvat vs musiikki)

\subsection{Epäkaupallisuuden epämääräisyys}

missä menee kaupallisuuden raja? onko pirate bay kaupallinen?
mainostuloja? saako ylläpitokulut kattaa?

%http://wiki.creativecommons.org/Defining_Noncommercial

cc-nc

\subsection{Teokset muuttuvat suljetuiksi}

vrt patentit

tietokoneohjelmat ja -pelit pilveen

\subsection{Valvonnan vaikeus ei tee lailliseksi}

\section{Muita ehdotuksia}

\subsection{Hyvitysmaksun poistaminen}

Maksua peritään muustakin käytöstä

Ei kohdistu oikein

\subsection{Kopiointisuojausten poistaminen}

Tekijöiden mielestä CD/DVD/Bluray-levyä myytäessä ei varsinaisesti
myydä fyysistä levyä vaan sillä olevaa informaatiota, joten nimenomaan
kyseisen informaation eteenpäin jakaminen ja kopioiminen on
kiellettyä.  Mutta jos kerran myynnin kohteena on informaatio, tulisi
myyjän taata, että kuluttaja pystyy lukemaan sen informaation (ja
säilyttämään sitä missä muodossa haluaa).  Eli sen sijaan että
kuluttajilla ei ole \emph{oikeutta} kiertää kopiointisuojausta,
myyjillä pitäisi olla \emph{velvollisuus} antaa myymänsä informaatio
ilman kopiointisuojausta tai muita estoja.

kopiosuojauksen poisto vähentää myyntiä

musiikin osalta kopiosuojaukset pääosin poistettu

kopiosuojaus haittaa laillisia käyttäjiä, ei estä piraatteja

Lisäksi DRM \footnote{effi} Kuka saa kuluttaa, missä ja milloin
``Käyttöestojen hallinnasta tuntuukin vain olevan haittaa maksaville
asiakkaille: musiikin kuuntelu vaikeutuu ja musiikin saa joka
tapauksessa halutessaan suojaamattomana.''

Scientific American

\subsection{Suoja-ajan lyhentäminen}

mahdollisimman lyhyt mutta tarpeeksi pitkä

\subsection{Parempia palveluita}

\subsection{Avoimen sisällön tuottaminen}

cc-by-sa gpl

saadaan luotua nykyisiä pykäliä hyödyntämällä oma vapaa maailma

Mikäli laillisesti on ilmaiseksi saatavilla riittävä valikoima
riittävän laadukasta sisältöä, maksullinen sisältö joutuu kriisiin.

julkinen data (esim. karttapalvelut)

\section{Yhteenveto}

Yksityisen kopioinnin salliminen ongelmallista

Muilla korjaustoimenpiteillä saavutetaan jo merkittäviä hyötyjä.

% Lähdeluettelo
\pagebreak
\bibliography{viitteet}

\end{document}

%%% Local Variables: 
%%% mode: latex
%%% TeX-master: t
%%% End: 
