\documentclass[titlepage,12pt]{article}

% Perussäädöt
\usepackage[T1]{fontenc}
\usepackage[utf8]{inputenc}
\usepackage[finnish]{babel}

\usepackage{url}

% Fontti: Linux Libertine
\usepackage{libertine}

% Viitteiden hallinta
\usepackage[authoryear,round]{natbib}
\usepackage[nottoc]{tocbibind}

% Otsikkosivun modaus
\usepackage{titling}
\preauthor{
  \vfill \large
  \begin{flushright}
    }

    \postauthor{
  \end{flushright}
}

% Abbreviations
%\usepackage[⟨options ⟩]{nomencl}
%\makenomenclature

% Riviväli 1.5
\usepackage{setspace}
\onehalfspacing

% Otsikkosivun tiedot
\title{Tekijänoikeudet}
\author{
  Seminaarityö\\
  \hfill Syksy 2012\\
  Jaakko Luttinen\\
  $62838$F\\
  Opponentti: N.N.
}
\date{ }



\begin{document}

% Viitteiden asetuksia. Babelin takia nämä pitää olla \begin{document}
% jälkeen.
\renewcommand\refname{Lähteet}
\renewcommand\bibname{Lähteet}
\bibliographystyle{plainnat}
\nocite{*}

% Lado otsikkosivu
\maketitle

% Sisällysluettelo
\tableofcontents

%\section*{Lyhenteet}


\pagebreak


\section{Johdanto}


Laillisella yksityisellä kopioinnilla tarkoitetaan tekijänoikeuslain
(TekL) mukaan sitä, että julkistetusta teoksesta saa kuka tahansa
tehdä muutaman kopion yksityistä käyttöään varten (TekL 12 §). Teoksen
jakaminen ei ole sallittua ilman tekijän lupaa, mikäli
tekijänoikeudellinen suoja on voimassa.  Esimerkiksi internetin
välityksellä tapahtuukin valtava määrä kopiointia yksityistä käyttöä
varten mutta laittomasti, sillä teosta ei ole laillisesti saatettu
jaettavaksi.

%Yksityisen kopioinnin lyhyt historia.

%Mitä yksityinen kopiointi tällä hetkellä on?

Piraattipuolue esittää, että kaikenlainen yksityiseen käyttöön
tapahtuva epäkaupallinen kopiointi tulisi
laillistaa.\footnote{Ks. \url{http://piraattipuolue.fi} / Agenda /
  Tekijänoikeus / Yksityinen kopiointi, 30.11.2012} Luvussa~2
esitellään ehdotuksen perusteluja, jotka voidaan tiivistää siihen,
ettei kopioinnista aiheudu tekijöille haittaa mutta kuluttajat
hyötyisivät, joten yhteiskunta kokonaisuutena hyötyisi merkittävästi.
Luku~3 esittää kritiikkiä ehdotusta kohtaan väittämällä, että siitä
kärsisivät sekä tekijät (eivät saisi korvausta) että kuluttajat (eivät
saisi teoksia ilman tekijöitä).  Luku~4 esitteleekin lyhyesti muutamia
muita korjauksia tekijänoikeuslakiin.

%Mitä piraattipuolue haluaa yksityiseksi kopioinniksi?

%Millä perusteilla?

%Mitä vikaa näissä perusteissa on?

%Olisiko muita ratkaisuja?

\section{Piraattipuolueen ehdotuksen perustelut}

Tämä luku esittelee Piraattipuolueen perusteluja kaiken yksityisen
kopioinnin sallimiseksi.  Perusteluja on koottu sekä puolueen omasta
esityksestä\footnote{Ks. \url{http://piraattipuolue.fi} / Agenda /
  Tekijänoikeus / Yksityinen kopiointi, 30.11.2012} että
Jokapiraatin\-oikeus-kirjasta, joka ei varsinaisesti ole puolueen
virallinen kannanotto mutta johon puolue itse kuitenkin viittaa
argumenteissaan.

\subsection{Ei taloudellista haittaa tekijöille}

Yksityisessä kopioinnissa ei ole kyse varastamisesta perinteisessä
mielessä, koska kopioidessa teos ei poistu alkuperäiseltä omistajalta.
Tekijät kuitenkin väittävät, että haitta ilmenee vähenevänä myyntinä,
koska kuluttajat voivat hankkia teoksen ilmaiseksi kopioimalla.
Piraattipuolueen mukaan tämä ei kuitenkaan pidä paikkaansa.

Ensinnäkin kulutustottumukset ovat ylipäänsä muuttuneet (musiikin
käyttö vähentynyt, internetin selailu lisääntynyt) ja ilmaistuotanto
tarjoaa laillisen ilmaisen vaihtoehdon maksulliselle
sisällölle.\footnote{Ks. Jokapiraatinoikeus s. 65, 94} Toisekseen
kuluttajat saattavat ostaa teoksen joka tapauksessa, jos pitävät
siitä, ja ilmaiseksi hyödyntävien aiheuttamat tappiot korvautuvat
niillä uusilla ostajilla, jotka eivät ilman ilmaista versiota olisi
kokeilleet ja päätyneet ostamaan
teosta.\footnote{Ks. \url{http://piraattipuolue.fi} / Agenda /
  Tekijänoikeus / Yksityinen kopiointi, 30.11.2012}

Laittoman kopioinnin vaikutuksia myyntilukuihin on tutkittu
runsaasti.\footnote{Ks. Jokapiraatinoikeus, s. 83--85} Arviot
myynnin vähenemisestä vaihtelevat muutamasta prosentista jopa 30
prosenttiin.  Mutta kuten todettiin yllä, myynnin vähentymiseen voi
vaikuttaa muitakin täysin laillisia tekijöitä.  Kun on tutkittu
suoraan tiedostonjakoverkkojen käytön vaikutusta myyntiin, ei ole
havaittu mitään vaikutusta.%\footnote{83-85}


\subsection{Yhteiskunta hyötyy}

%Nolla kustannus, positiivinen hyöty

%Jos kaikille voidaan antaa karkki ilmaiseksi, niin eikö se kannata?

Teokset ovat kulttuurista pääomaa, jotka lisäävät ihmisten
hyvinvointia.  Jokainen ihminen siis saa hyötyä teoksista joihin
pääsee käsiksi ja täten myös koko yhteiskunta hyötyy yksittäisten
ihmisten joukkona.  Jos nämä teokset ovat digitaalisessa muodossa,
niitä voidaan jakaa käytännössä kaikille ihmisille olemattomalla
kustannuksella.  Kun kerran on mahdollista lisätä koko kansan
hyvinvointia ilman kustannuksia, eikö se olisi yhteiskunnalle järkevä
tapa toimia?  Jos yksi leipä voitaisiin ilman lisäkustannuksia
kopioida kaikkien ruuaksi, eikö se olisi kannattavaa?

``Varastamisessa on kyse siitä, että jokin esine, jolla on arvoa,
vaihtaa omistajaa.  Yhteiskunnan elintaso ei kohoa, mutta varkauden
uhrin elintaso laskee. Kopioinnissa asia, jolla on arvoa, monistuu.
Yhteiskunnan elintaso kohoaa, koska yhä useammilla on yhä enemmän
arvokkaita asioita hallussaan.  Kopioinnissa ei ole uhria, jonka
elintaso laskisi.''\footnote{Ks. \url{http://piraattipuolue.fi} /
  Agenda / Tekijänoikeus / Yksityinen kopiointi, 30.11.2012}

\subsection{Kuluttajien yksityisyys säilyy}

Laittoman yksityisen kopioinnin valvonta uhkaa viestintäsalaisuutta ja
oikeusturvaa.  Koska kopiointi tapahtuu muun verkkoliikenteen
joukossa, tehokas valvonta vaatisi kaiken verkkoliikenteen
seuraamista, jolloin yksityisyys olisi uhattuna.  Valvonnan
tehostaminen todennäköisimmin vain johtaisi tiukempaan
verkkoliikenteen salaamiseen ja
anonymisointiin.\footnote{\url{http://piraattipuolue.fi} / Agenda /
  Tekijänoikeus / Yksityinen kopiointi, 30.11.2012} 

Toisaalta toimintaa voidaan koittaa ehkäistä sensuroimalla laittomassa
kopioinnissa hyödynnettäviä sivustoja. %\footnote{PirateBay Elisa}
Tällaiset sensuurit ovat kuitenkin sekä tehottomia että sananvapautta
loukkaavia.\footnote{Ks. Piraattipuolueen lehdistötiedotteet 3.10.2011
  ja 26.10.2011.}
%\url{http://piraattipuolue.fi/2011/10/piraattipuolue-tuomitsee-raesaesen-puheet-nettisensuurin-laajentamisesta}}
%\url{http://piraattipuolue.fi/2011/10/the-pirate-bay-sensuuri-uhka-sananvapaudelle}

\subsection{Tekijöille näkyvyyttä}

Digitaalisten teosten leviäminen ilmaiseksi kopioimalla voidaan nähdä
myös tekijöiden kannalta hyödyllisenä.  Kuluttajat voivat kokeilla
teosta ja ostaa sen, mikäli pitävät.  Ilman kokeilumahdollisuutta
kuluttaja ei olisi välttämättä päätynyt tekemään ostosta.  Tällä
tavalla erityisesti vähemmän tunnetut tekijät saavat markkinoitua
itseään tehokkaasti.  Vaikka kuluttaja ei ostaisikaan ilmaiseksi
lataamaansa tuotetta, kopiointi voi silti koitua tekijän hyödyksi,
mikäli kopioija päätyykin esimerkiksi ostamaan fanituotteita tai lipun
keikalle.


\section{Piraattipuolueen ehdotuksen kritiikki}

Tämä luku esittää kirjoittajan kritiikkiä Piraattipuolueen esitystä
kohtaan.  Keskeisenä huolena on se, että kaiken yksityisen kopioinnin
laillistaminen johtaisi tekijöiden puutteeseen ja teosten entistäkin
tiukempaan suojaamiseen, mikä ei ole kuluttajan kannalta hyvä
lopputulos.

\subsection{Riittämättömät liiketoimintamallit}

Yksi merkittävä uhka Piraattipuolueen määrittämän yksityisen
kopioinnin sallimisessa on luovan työn muuttuminen taloudellisesti
kannattamattomaksi.  Yhteiskunnan (ja yksilöiden) kannalta on parasta,
kun mahdollisimman suurta hyötyä voidaan jakaa mahdollisimman monelle.
Yksityinen kopiointi mahdollistaa mahdollisimman monelle jakamisen,
mutta mikäli sen seurauksena teosten laatu ja määrä laskisi (eli hyöty
pienenisi), voi yhteiskunnan näkökulmasta suurin kokonaishyöty olla
saavutettavissa, kun mahdollistetaan tekijöille kannattavia
liiketoimintamalleja laadukkaiden teoksien tuottamiseksi.  Näihin
kysymyksiin Piraattipuolue ottaa kantaa kovin
ylimalkaisesti.\footnote{Ks. Jokapiraatinoikeus s. 97--101}

% Mikä malli ikinä olisikaan, se on siinä mielessä tasapuolinen, että
% jokaisella ihmisellä kuitenkin on aivan yhtäläinen mahdollisuus
% valita, ryhtyykö luovalle alalle.

% Lisäksi paljon hoettu toteamus ``tekijällä on oikeus saada korvaus''
% ei pidä paikkaansa, vaan paremminkin pitäisi sanoa, että "tekijällä
% on oikeus pyytää korvausta".

Yksi ehdotettu rahoitusmuoto olisi ihmisten vapaaehtoiset
maksut.\footnote{Ks. Jokapiraatinoikeus s. 90} Vaikka kuluttajat
lataisivat teokset ilmaiseksi, he maksaisivat tekijälle tai ostaisivat
teoksen maksullisen version, mikäli pitävät teoksesta.  Tähän
kuitenkin liittyy paljon ongelmia: Mikäli teoksen voi saada
ilmaiseksi, tekijällä ei ole muuta rationaalista syytä maksaa
teoksesta kuin se, että saisi vastaavia teoksia jatkossakin.
Toisaalta yksittäisen ihmisen maksun merkitys on niin olematon, että
tuleville teoksille joko on tai ei ole riittävästi taloudellista tukea
tämän yksittäisen kuluttajan päätöksestä riippumatta.  Täten
kuluttajalla ei useimmiten ole syytä maksaa ilmaisesta teoksesta --
ainakaan sellaista summaa, jonka olisi valmis maksamaan, jos teosta ei
muuten saisi.  Ja vaikka ihmiset ostaisivatkin teoksen ensin
kokeiltuaan sitä ilmaiseksi, tämä malli ei toimisi luonteeltaan
kertakäyttöisillä teoksilla kuten elokuvilla.

%jokapiraatinoikeus s.90, ihmiset maksaisivat koska se on hyvä tapa..


Toinen lähestymistapa on tehdä liiketoimintaa muulla kuin itse teosten
myynnillä.  Varsinainen myytävä tuote voi olla palveluita
(esim. bändin keikat) tai fyysisiä tuotteita (esim. Angry Birds
-lelut).  Ongelmana tässä voi olla päätoiminnasta eli teosten
luomisesta tuleekin sivutoimintaa.  Toisaalta toimintaa voi koittaa
tukea rahoittajien avulla, eli tekijä hankkii ensin riittävä määrän
rahoittajia ja vasta sen jälkeen luo teoksen (esim. Kickstarterin
kautta rahoitus).  Tähän puolestaan liittyy edellisessä kappaleessa
kuvattuja ongelmia.  Yhteiskunta voisi tukea tekijöitä suoraan
verovaroista, mutta tällöin päättäjät voivat kohdistaa tukensa
epäreilusti tai ``vääriin'' kohteisiin.\footnote{Ks. Jokapiraatinoikeus
  s. 214, 246}


% Palvelut tai fyysiset tuotteet varsinaisena myytävänä

% Rahoittajat ja tilaajat

% Lahjoitukset ja fanituotteet


% kokeilumalli toimii vain jos teosta hyödynnetään useita kertoja
% (elokuvat vs musiikki)

\subsection{Epäkaupallisuuden epämääräisyys}

Piraattipuolue haluaa rajoittaa yksityisen kopioinnin epäkaupalliseen
toimintaan.  Terminä epäkaupallinen on kuitenkin sen verran
epämääräinen, että epäselviä tapauksia olisi valtava määrä.  Saako
teoksia jakamalla houkutella kuluttajia omille nettisivuilleen, jossa
heidän avullaan tienataan muilla tavoin? Entä jos tienataan vain
ylläpitokustannuksien verran?  Entä jos ylläpitokustannuksiin sisältyy
henkilötyötunteja?  Entä jos nettisivut ovatkin voittoa
tavoittelemattoman organisaation?  Saako kaupallinen yritys hyödyntää
teosta, jos se ei suoraan rahasta sillä?

Creative Commons suoritti tutkimuksen siitä, miten ihmiset ymmärtävät
epäkaupallisuuden.\footnote{Ks. Creative Commons 2009, 30.11.2012}
Sekä tekijät että kuluttajat ymmärsivät sen suurin piirtein samalla
tavalla ja usein tekijät itse ajattelivat rajoituksen väljempänä kuin
kuluttajat.  Joka tapauksessa epäkaupallisuuden riittävän täsmällinen
määritteleminen olisi välttämätöntä ja se mitä todennäköisimmin
sisältäisi paljon ns. ``porsaanreikiä'', joiden avulla teoksia
pyrittäisiin hyödyntämään lain mukaisesti mutta taloudellista hyötyä
epäsuorasti saavuttaen.

% ``The one exception to this pattern is in relation to uses by
% individuals that are personal or private in nature .  Here, it is
% users (not creators) who believe such uses are less commercial''


%missä menee kaupallisuuden raja? onko pirate bay kaupallinen?
%mainostuloja? saako ylläpitokulut kattaa?

%http://wiki.creativecommons.org/Defining_Noncommercial

%cc-nc

\subsection{Teokset muuttuvat suljetuiksi}

Mikäli teoksilla ei enää olisi riittävää tekijänoikeudellista suojaa,
ne voisivat muuttua suljetummiksi.  Jos tilannetta verrataan
patentteihin, niin patenttijärjestelmän purkaminen mitä
todennäköisimmin johtaisi keksintöjen salailuun ja teknisten
ratkaisujen piilotteluun.  Mikäli yksityiseen käyttöön kopiointia ei
kielletä, esimerkiksi tietokonepelit ja -ohjelmistot siirtyisivät
todennäköisesti vieläkin enemmän maksullisiksi pilvipalveluiksi.
Tällöin kuluttajille ei milloinkaan anneta itse ohjelmaa, jota he
voisivat levittää.  Ei-interaktiivisten teosten kuten musiikin ja
elokuvien osalta tällainen suojaaminen olisi vaikeampaa, sillä kuvan
ja äänen kaappaaminen on teknisesti helppoa, mutta tällaista suojausta
pyritään toteuttamaan jo nykyisillä teknisillä kopiosuojauksilla ja
DRM:llä.  Joka tapauksessa tietokoneohjelmien siirtyminen
pilvipalveluiksi ja kopiosuojausten tehostaminen ei ole kuluttajien
kannalta läheskään aina hyvä asia, joten kuluttajat voisivat
kokonaisuudessaan kärsiä kopioinnin sallimisesta.

%vrt patentit

%tietokoneohjelmat ja -pelit pilveen

\subsection{Valvonnan vaikeus ei tee lailliseksi}

Vaikka laittoman lataamisen valvonta onkin vaikeaa, ei siitä tulisi
päätellä, että yksityinen kopiointi täytyy laillistaa.  Monien
muidenkin laittomien asioiden tehokas valvonta rikkoisi ihmisten
yksityisyyttä, mutta niitä pidetään silti laittomina.  Valvontaa
pitääkin pyrkiä kohdistamaan yksittäisten pienkäyttäjien sijaan
suuriin toimijoihin.  Lataaminen voi toki siirtyä entistä enemmän
suojattuihin anonyymeihin verkkoihin, jolloin rikollisten
jäljittäminen vaikeutuu huomattavasti.  Mutta samalla tavalla näissä
verkoissa on mahdollista suorittaa muutakin laitonta toimintaa kuten
huumeiden tai varastetun tavaran myyntiä tai sananvapauden rajat
rikkovaa kirjoittelua, uhkailua ja syyttelyä.  Yleisellä tasolla onkin
erittäin mielenkiintoinen ja tärkeä kysymys, voidaanko pitää
laittomana sellaista asiaa, jota ei pystytä valvomaan tai estämään,
mutta kantoja ei tulisi kohdistaa valikoiden tekijänoikeuskysymyksiin.


\section{Muita ehdotuksia}

Tässä luvussa esitellään lyhyesti muutama muu keino parantaa
kuluttajien oikeuksia ilman, että tekijöiden suoja ja taloudellinen
asema merkittävästi heikkenee.  Ehdotukset ovat kirjoittajan eivätkä
Piraattipuolueen, mutta näissä kysymyksissä näkemykset ovat melko
samanlaisia.

\subsection{Suoja-ajan lyhentäminen}

Tekijänoikeuden suoja-ajan tulee olla mahdollisimman lyhyt, jotta
yhteiskunta voi rikastua kulttuurin hyödyntämisestä, mutta riittävän
pitkä, jotta laadukkaiden teosten tekijöitä silti löytyisi.  Tällä
hetkellä tekijänoikeuden suoja-aika on 70 vuotta tekijän kuolemasta
(TekL § 43).  Koska merkittävä määrä teoksen tuottamista tuloista
tulee muutaman ensimmäisen vuoden aikana, olisi perusteltua lyhentää
suoja-aikaa esimerkiksi 5--10 vuoteen.\footnote{Ks. Jokapiraatinoikeus
  s. 204}


\subsection{Kopiointisuojausten poistaminen}

Tekijöiden mielestä CD/DVD/Bluray-levyä myytäessä ei varsinaisesti
myydä fyysistä levyä vaan sillä olevaa informaatiota, joten nimenomaan
kyseisen informaation eteenpäin jakaminen ja kopioiminen on
kiellettyä.  Mutta jos kerran myynnin kohteena on informaatio, tulisi
myyjän taata, että kuluttajalla on mahdollisuus lukea se informaatio
(ja säilyttää sitä missä muodossa haluaa).  Eli sen sijaan että
kuluttajilla ei ole \emph{oikeutta} kiertää kopiointisuojausta,
myyjillä pitäisi olla \emph{velvollisuus} antaa myymänsä informaatio
ilman kopiointisuojausta tai muita estoja.
%
%kopiosuojauksen poisto vähentää myyntiä
%
%musiikin osalta kopiosuojaukset pääosin poistettu
%
%kopiosuojaus haittaa laillisia käyttäjiä, ei estä piraatteja
%
Käyttöoikeuksien hallintakin voi rajoittaa kuka, missä ja milloin voi
teosta hyödyntää.  ``Käyttöestojen hallinnasta tuntuukin vain olevan
haittaa maksaville asiakkaille: musiikin kuuntelu vaikeutuu ja
musiikin saa joka tapauksessa halutessaan
suojaamattomana.''\footnote{Ks. \url{http://www.effi.org/tekijanoikeus/aanitteet/drm.html},
  30.11.2012} Tämän vuoksi estot poistamalla palveltaisiin ennen
kaikkea lainkuuliaisia kuluttajia ja tekijöitä.

%Scientific American


%\subsection{Parempia palveluita}

\subsection{Hyvitysmaksun poistaminen}

Hyvitysmaksulla kompensoidaan yksityisestä kopioinnista aiheutuvaa
haittaa perimällä erilaisista dataa tallentavista laitteista pieni
maksu.  Se ei kuitenkaan kohdistu oikein kuluttajiin, sillä kaikki
joutuvat sen maksamaan, vaikka säilyttäisivät vain omia kuviaan ja
videoitaan.  Se ei kohdistu oikein myöskään tekijöille, sillä opetus-
ja kulttuuriministeriö päättää rahan jaosta, jolla ei välttämättä ole
mitään tekemistä sen kanssa, keiden teoksia ihmisillä on yksityisesti
kopioituina.  Hyvitysmaksusta olisikin siis luovuttava ja mahdolliset
välttämättömät tuet otettava suoraan valtion budjetista.
\footnote{Ks. \url{http://blogi.piraattipuolue.fi/2011/01/11/},
  30.11.2012}


\subsection{Avoimen sisällön tuottaminen}

Tälläkin hetkellä sisällöntuotanto kamppailee laillisesti ilmaiseksi
jaossa olevien teosten kanssa.\footnote{Ks. Jokapiraatinoikeus s. 91}
Ainoa keino pärjätä tällaisessa tilanteessa voi olla teosten jakaminen
ilmaiseksi itsekin.  Tarttuvilla avoimilla lisensseillä
(esim. CC-BY-SA ja GPL) saadaan jopa luotua tavallaan oma erillinen
maailmansa, jossa teoksia saa vapaasti käyttää ja jakaa kunhan sallii
sen muillekin.  Tämä sisältömaailma on erillinen suljettujen teosten
maailmasta (kaupallisuudella ei ole merkitystä), joten tämä
suljettujen teosten maailma voi ajan kuluessa näivettyä, jos avointen
lisenssien maailmaan tulee riittävän kattava ja laadukas valikoima.
Tämä idealistinen avoin maailma perustuu hieman paradoksaalisesti
nykyisiin tekijänoikeuspykäliin, joten mitään lakimuutoksia tämän
osalta ei edes tarvita -- ainoastaan tekijöiden halu.

% cc-by-sa gpl

% saadaan luotua nykyisiä pykäliä hyödyntämällä oma vapaa maailma

% Mikäli laillisesti on ilmaiseksi saatavilla riittävä valikoima
% riittävän laadukasta sisältöä, maksullinen sisältö joutuu kriisiin.

% julkinen data (esim. karttapalvelut)

\section{Yhteenveto}

Raportti tarkasteli Piraattipuolueen ehdotusta kaiken yksityisen
kopioinnin sallimiseksi.  Piraattipuolueen radikaalin ehdotuksen
argumentit vaikuttivat kuitenkin hieman riittämättömiltä verrattuna
mahdollisiin haittoihin.  Tämän takia raportti esitti muita
ehdotuksia, joilla kuluttajien asemaa voitaisiin parantaa ilman, että
tekijänoikeuden suojaa tarvitsee yksityisen käytön osalta kokonaan
poistaa.  Näillä ehdotuksilla (esim. suoja-ajan lyhentämisellä) voisi
olla suuri merkitys yhteiskunnan kokonaishyödyn kasvattamisessa.

%Yksityisen kopioinnin salliminen ongelmallista

%Muilla korjaustoimenpiteillä saavutetaan jo merkittäviä hyötyjä.

% Lähdeluettelo
\pagebreak
\bibliography{viitteet}
% \section*{Lähteet}

% \begin{thebibliography}
  
% \bibitem{lehdisto1}
%   huh%\url{http://piraattipuolue.fi/2011/10/piraattipuolue-tuomitsee-raesaesen-puheet-nettisensuurin-laajentamisesta}.

% \bibitem{lehdisto2}
%   hah%\url{http://piraattipuolue.fi/2011/10/the-pirate-bay-sensuuri-uhka-sananvapaudelle}.
    
% \end{thebibliography}

\end{document}

%%% Local Variables: 
%%% mode: latex
%%% TeX-master: t
%%% End: 
