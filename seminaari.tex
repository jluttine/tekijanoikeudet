\documentclass[titlepage,12pt]{article}

% Perussäädöt
\usepackage[T1]{fontenc}
\usepackage[utf8]{inputenc}
\usepackage[finnish]{babel}

\usepackage{url}

% Fontti: Linux Libertine
\usepackage{libertine}

% Viitteiden hallinta
\usepackage[authoryear,round,sort]{natbib}
\usepackage[nottoc]{tocbibind}

% Otsikkosivun modaus
\usepackage{titling}
\preauthor{
  \vfill \large
  \begin{flushright}
    }

    \postauthor{
  \end{flushright}
}

% Abbreviations
%\usepackage[⟨options ⟩]{nomencl}
%\makenomenclature

% Riviväli 1.5
\usepackage{setspace}
\onehalfspacing

% Otsikkosivun tiedot
\title{Piraattipuolueen ehdotus kaiken yksityisen kopioinnin
  laillistamiseksi\\ --- \\ Kriittinen arvio}
\author{
  Seminaarityö\\
  \hfill Syksy 2012\\
  Jaakko Luttinen\\
  $62838$F\\
  Opponentti: N.N.
}
\date{ }



\begin{document}

% Viitteiden asetuksia. Babelin takia nämä pitää olla \begin{document}
% jälkeen.
\renewcommand\refname{Lähteet}
\renewcommand\bibname{Lähteet}
\bibliographystyle{plainnat}
\nocite{*}

% Lado otsikkosivu
\maketitle

% Sisällysluettelo
\tableofcontents

%\section*{Lyhenteet}


\pagebreak


\section{Johdanto}


Laillisella yksityisellä kopioinnilla tarkoitetaan tekijänoikeuslain
(TekL) mukaan sitä, että julkistetusta teoksesta saa kuka tahansa
tehdä muutaman kopion yksityistä käyttöään varten (TekL 12 §). Teoksen
jakaminen ei ole sallittua ilman tekijän lupaa, mikäli
tekijänoikeudellinen suoja on voimassa.  Esimerkiksi internetin
välityksellä tapahtuukin valtava määrä kopiointia yksityistä käyttöä
varten mutta laittomasti, sillä teosta ei ole laillisesti saatettu
jaettavaksi.

%Yleistä höpinää yksityisestä kopioinnista ongelmista ratkaisuista

%Yksityisen kopioinnin lyhyt historia.

%Mitä yksityinen kopiointi tällä hetkellä on?

Piraattipuolue esittää, että kaikenlainen yksityiseen käyttöön
tapahtuva epäkaupallinen kopiointi tulisi
laillistaa.\footnote{Ks. \url{http://piraattipuolue.fi} / Agenda /
  Tekijänoikeus / Yksityinen kopiointi, 30.11.2012} Luvussa~2
esitellään ehdotuksen perusteluja, jotka voidaan tiivistää siihen,
ettei kopioinnista aiheudu tekijöille haittaa mutta kuluttajat
hyötyisivät, joten yhteiskunta kokonaisuutena hyötyisi merkittävästi.
Luvussa~3 esitetään kritiikkiä ehdotusta kohtaan väittämällä, että
siitä kärsisivät sekä tekijät, jotka eivät saisi korvausta, että
kuluttajat, jotka eivät saisi teoksia ilman tekijöitä.  Luvussa~4
esitelläänkin lyhyesti muutamia muita muutosehdotuksia
tekijänoikeuslakiin.

%Mitä piraattipuolue haluaa yksityiseksi kopioinniksi?

%Millä perusteilla?

%Mitä vikaa näissä perusteissa on?

%Olisiko muita ratkaisuja?

\section{Piraattipuolueen perustelut yksityisen kopioinnin
  laillistamiselle}

Tässä luvussa esitellään piraattipuolueen perusteluja kaiken
yksityisen kopioinnin sallimiseksi.  Perusteluja on koottu sekä
puolueen omasta
esityksestä\footnote{Ks. \url{http://piraattipuolue.fi} / Agenda /
  Tekijänoikeus / Yksityinen kopiointi, 30.11.2012} että
Jokapiraatin\-oikeus-kirjasta, joka ei varsinaisesti ole puolueen
virallinen kannanotto mutta johon puolue itse kuitenkin viittaa
perusteluissaan.

\subsection{Taloudellinen haitattomuus tekijöille}

Yksityisessä kopioinnissa ei ole kyse varastamisesta perinteisessä
mielessä, koska kopioidessa teos ei poistu alkuperäiseltä omistajalta.
Teosten tekijöiden mukaan haitta ilmenee vähenevänä myyntinä, koska
kuluttajat voivat hankkia teoksen ilmaiseksi kopioimalla.
Piraattipuolueen mukaan tämä ei kuitenkaan pidä paikkaansa.
Ensinnäkin kulutustottumukset ovat ylipäänsä muuttuneet: musiikin
käyttö vähentynyt, internetin selailu lisääntynyt, ja ilmaistuotanto
tarjoaa laillisen ilmaisen vaihtoehdon maksulliselle
sisällölle.\footnote{Ks. Jokapiraatinoikeus s. 65, 94} Toisekseen
kuluttajat saattavat ostaa teoksen joka tapauksessa, jos pitävät
siitä, ja ilmaiseksi teosta hyödyntävien aiheuttamat tappiot
korvautuvat niillä uusilla ostajilla, jotka eivät ilman ilmaista
versiota olisi kokeilleet ja päätyneet ostamaan
teosta.\footnote{Ks. \url{http://piraattipuolue.fi} / Agenda /
  Tekijänoikeus / Yksityinen kopiointi, 30.11.2012}

Laittoman kopioinnin vaikutuksia myyntilukuihin on tutkittu
runsaasti.
%\footnote{Ks. Jokapiraatinoikeus, s. 83--85}
Arviot myynnin vähenemisestä vaihtelevat muutamasta prosentista jopa
30 prosenttiin, mutta kuten todettiin yllä, myynnin vähentymiseen voi
vaikuttaa muitakin täysin laillisia tekijöitä.  Kun on tutkittu
suoraan tiedostonjakoverkkojen käytön vaikutusta myyntiin, ei ole
havaittu mitään vaikutusta. \footnote{Ks. Jokapiraatinoikeus
  s. 83--85}


\subsection{Yhteiskunnallinen hyöty}

%Nolla kustannus, positiivinen hyöty

%Jos kaikille voidaan antaa karkki ilmaiseksi, niin eikö se kannata?

Teokset ovat piraattipuolueen mukaan kulttuurista pääomaa, jotka
lisäävät ihmisten hyvinvointia.  Jokainen ihminen siis saa hyötyä
teoksista joihin pääsee käsiksi ja täten myös koko yhteiskunta hyötyy
yksittäisten ihmisten joukkona.  Jos nämä teokset ovat digitaalisessa
muodossa, niitä voidaan jakaa käytännössä kaikille ihmisille
olemattomalla kustannuksella.  Kun kerran on mahdollista lisätä koko
kansan hyvinvointia ilman kustannuksia, eikö se olisi yhteiskunnalle
järkevä tapa toimia?  Jos yksi leipä voitaisiin ilman lisäkustannuksia
kopioida kaikkien ruuaksi, eikö se olisi kannattavaa?

``Varastamisessa on kyse siitä, että jokin esine, jolla on arvoa,
vaihtaa omistajaa.  Yhteiskunnan elintaso ei kohoa, mutta varkauden
uhrin elintaso laskee. Kopioinnissa asia, jolla on arvoa, monistuu.
Yhteiskunnan elintaso kohoaa, koska yhä useammilla on yhä enemmän
arvokkaita asioita hallussaan.  Kopioinnissa ei ole uhria, jonka
elintaso laskisi.'' \footnote{Ks. \url{http://piraattipuolue.fi} /
  Agenda / Tekijänoikeus / Yksityinen kopiointi, 30.11.2012}

\subsection{Kuluttajien yksityisyyden säilyminen}

Laittoman yksityisen kopioinnin valvonta uhkaa piraattipuolueen mukaan
viestintäsalaisuutta ja oikeusturvaa.  Koska kopiointi tapahtuu muun
verkkoliikenteen joukossa, tehokas valvonta vaatisi kaiken
verkkoliikenteen seuraamista, jolloin kuluttajan yksityisyys olisi
uhattuna.  Valvonnan tehostaminen todennäköisimmin vain johtaisi
tiukempaan verkkoliikenteen salaamiseen ja
anonymisointiin.\footnote{\url{http://piraattipuolue.fi} / Agenda /
  Tekijänoikeus / Yksityinen kopiointi, 30.11.2012}

Toisaalta toimintaa voidaan koittaa ehkäistä sensuroimalla laittomassa
kopioinnissa hyödynnettäviä sivustoja. %\footnote{PirateBay Elisa}
Tällaiset sensuurit ovat kuitenkin sekä tehottomia että sananvapautta
loukkaavia.\footnote{Ks. piraattipuolueen lehdistötiedotteet 3.10.2011
  ja 26.10.2011.}
%\url{http://piraattipuolue.fi/2011/10/piraattipuolue-tuomitsee-raesaesen-puheet-nettisensuurin-laajentamisesta}}
%\url{http://piraattipuolue.fi/2011/10/the-pirate-bay-sensuuri-uhka-sananvapaudelle}

\subsection{Tekijöiden näkyvyys}

Digitaalisten teosten leviäminen ilmaiseksi kopioimalla voidaan
piraattipuolueen mukaan nähdä myös tekijöiden kannalta hyödyllisenä.
Kuluttajat voivat kokeilla teosta ja ostaa sen, mikäli pitävät siitä.
Ilman kokeilumahdollisuutta kuluttaja ei olisi välttämättä päätynyt
tekemään ostosta.  Tällä tavalla erityisesti vähemmän tunnetut tekijät
voivat markkinoida itseään tehokkaasti.  Vaikka kuluttaja ei
ostaisikaan ilmaiseksi lataamaansa tuotetta, kopiointi voi silti
koitua tekijän hyödyksi, mikäli kopioija päätyykin esimerkiksi
ostamaan fanituotteita tai lipun keikalle.


\section{Piraattipuolueen ehdotuksen kritiikki}

Tässä luvussa esittelen kritiikkiäni piraattipuolueen esitystä
kohtaan.  Keskeinen ongelma piraattipuolueen ehdotuksessa on se, että
kaiken yksityisen kopioinnin laillistaminen johtaisi tekijöiden
puutteeseen, teosten laadun heikkenemiseen ja teosten entistäkin
tiukempaan suojaamiseen, mikä ei ole kuluttajan kannalta hyvä
lopputulos.

\subsection{Riittämättömät liiketoimintamallit}

Yksi merkittävä uhka piraattipuolueen ehdottamassa kaiken yksityisen
kopioinnin sallimisessa on luovan työn muuttuminen taloudellisesti
kannattamattomaksi.  Yhteiskunnan -- ja yksilöiden -- kannalta on
parasta, kun mahdollisimman suurta hyötyä voidaan jakaa mahdollisimman
monelle.  Yksityinen kopiointi mahdollistaa mahdollisimman monelle
jakamisen, mutta mikäli sen seurauksena teosten laatu ja määrä laskisi
eli hyöty pienenisi, voi yhteiskunnan näkökulmasta suurin
kokonaishyöty olla saavutettavissa, kun mahdollistetaan tekijöille
kannattavia liiketoimintamalleja laadukkaiden teoksien tuottamiseksi.
Näihin kysymyksiin piraattipuolue ottaa kantaa kovin
ylimalkaisesti.\footnote{Ks. Jokapiraatinoikeus s. 97--101}

% Mikä malli ikinä olisikaan, se on siinä mielessä tasapuolinen, että
% jokaisella ihmisellä kuitenkin on aivan yhtäläinen mahdollisuus
% valita, ryhtyykö luovalle alalle.

% Lisäksi paljon hoettu toteamus ``tekijällä on oikeus saada korvaus''
% ei pidä paikkaansa, vaan paremminkin pitäisi sanoa, että "tekijällä
% on oikeus pyytää korvausta".

Yksi ehdotettu rahoitusmuoto olisivat ihmisten vapaaehtoiset
maksut.\footnote{Ks. Jokapiraatinoikeus s. 90} Vaikka kuluttajat
lataisivat teokset ilmaiseksi, he maksaisivat tekijälle tai ostaisivat
teoksen maksullisen version, mikäli pitävät teoksesta.  Ongelmana
tässä on se, että mikäli teoksen voi saada ilmaiseksi, kuluttajalla ei
ole muuta rationaalista syytä maksaa teoksesta kuin se, että saisi
vastaavia teoksia jatkossakin.  Toisaalta yksittäisen ihmisen maksun
merkitys on niin olematon, että tuleville teoksille joko on tai ei ole
riittävästi taloudellista tukea tämän yksittäisen kuluttajan
päätöksestä riippumatta.  Täten kuluttajalla ei useimmiten ole syytä
maksaa ilmaisesta teoksesta -- ainakaan sellaista summaa, jonka olisi
valmis maksamaan, jos teosta ei muuten saisi, ja vaikka ihmiset
ostaisivatkin teoksen ensin kokeiltuaan sitä ilmaiseksi, tämä malli ei
toimisi luonteeltaan kertakäyttöisillä teoksilla kuten elokuvilla.

%jokapiraatinoikeus s.90, ihmiset maksaisivat koska se on hyvä tapa..


Toinen lähestymistapa on tehdä liiketoimintaa muulla kuin itse teosten
myynnillä.  Varsinainen myytävä tuote voi olla palveluita
(esim. bändin keikat) tai fyysisiä tuotteita (esim. Angry Birds
-lelut).  Ongelmana tässä voi olla, että päätoiminnasta eli teosten
luomisesta tuleekin sivutoimintaa.  Toisaalta toimintaa voi koittaa
tukea rahoittajien avulla, eli tekijä hankkii ensin riittävän määrän
rahoittajia ja vasta sen jälkeen luo teoksen (esim. Kickstarterin
kautta rahoitus).  Tähän puolestaan liittyy edellisessä kappaleessa
kuvattu ongelma eli kuluttajilla ei ole motivaatiota maksaa tuotteesta
niin paljon kuin olisivat valmiita maksamaan, jos teosta ei muuten
saisi.  Yhteiskunta voisi tukea tekijöitä suoraan verovaroista, mutta
tällöin päättäjät voivat kohdistaa tukensa epäreilusti tai ``vääriin''
kohteisiin.\footnote{Ks. Jokapiraatinoikeus s. 214, 246}


% Palvelut tai fyysiset tuotteet varsinaisena myytävänä

% Rahoittajat ja tilaajat

% Lahjoitukset ja fanituotteet


% kokeilumalli toimii vain jos teosta hyödynnetään useita kertoja
% (elokuvat vs musiikki)

\subsection{Epäkaupallisuuden epämääräisyys}

Piraattipuolue haluaa rajoittaa yksityisen kopioinnin epäkaupalliseen
toimintaan.  Terminä epäkaupallinen on kuitenkin sen verran
epämääräinen, että epäselviä tapauksia olisi valtava määrä.  Saako
teoksia jakamalla houkutella kuluttajia omille nettisivuilleen, jossa
heidän avullaan tienataan muilla tavoin? Entä jos tienataan vain
ylläpitokustannuksien verran?  Entä jos ylläpitokustannuksiin sisältyy
henkilötyötunteja?  Entä jos nettisivut ovatkin voittoa
tavoittelemattoman organisaation?  Saako kaupallinen yritys hyödyntää
teosta, jos se ei suoraan rahasta sillä?

Creative Commons suoritti tutkimuksen siitä, miten ihmiset ymmärtävät
epäkaupallisuuden.\footnote{Ks. Creative Commons 2009, 30.11.2012}
Sekä tekijät että kuluttajat ymmärsivät sen suurin piirtein samalla
tavalla ja usein tekijät itse ajattelivat rajoituksen väljempänä kuin
kuluttajat.  Joka tapauksessa epäkaupallisuuden riittävän täsmällinen
määritteleminen olisi välttämätöntä ja se mitä todennäköisimmin
sisältäisi paljon porsaanreikiä, joiden avulla teoksia olisi
mahdollista hyödyntää lain mukaisesti mutta taloudellista hyötyä
epäsuorasti saavuttaen.

% ``The one exception to this pattern is in relation to uses by
% individuals that are personal or private in nature .  Here, it is
% users (not creators) who believe such uses are less commercial''


%missä menee kaupallisuuden raja? onko pirate bay kaupallinen?
%mainostuloja? saako ylläpitokulut kattaa?

%http://wiki.creativecommons.org/Defining_Noncommercial

%cc-nc

\subsection{Teosten muuttuminen suljetuiksi}

Mikäli teoksilla ei enää olisi riittävää tekijänoikeudellista suojaa,
ne voisivat muuttua vielä nykyistäkin suljetummiksi.  Jos tilannetta
verrataan patentteihin, patenttijärjestelmän purkaminen mitä
todennäköisimmin johtaisi keksintöjen salailuun ja teknisten
ratkaisujen piilotteluun.  Mikäli yksityiseen käyttöön kopiointi ei
olisi kiellettyä, esimerkiksi tietokonepelit ja -ohjelmistot
siirtyisivät todennäköisesti vieläkin enemmän maksullisiksi
pilvipalveluiksi, mikä tarkoittaa sitä, että kuluttajille ei
milloinkaan anneta itse ohjelmaa, jota he voisivat levittää.
Ei-interaktiivisten teosten kuten musiikin ja elokuvien osalta
tällainen suojaaminen olisi vaikeampaa, sillä kuvan ja äänen
kaappaaminen on teknisesti helppoa, mutta tällaista suojausta pyritään
toteuttamaan jo nykyisillä teknisillä kopiosuojauksilla ja
käyttöoikeuksien hallinnalla.  Joka tapauksessa tietokoneohjelmien
siirtyminen pilvipalveluiksi ja kopiosuojausten tehostaminen ei ole
kuluttajien kannalta läheskään aina hyvä asia, joten kuluttajat
voisivat loppujen lopuksi kärsiä kopioinnin sallimisesta.

%vrt patentit

%tietokoneohjelmat ja -pelit pilveen

\subsection{Valvonnan vaikeus ei tee lailliseksi}

Vaikka laittoman lataamisen valvonta onkin vaikeaa, ei siitä tulisi
päätellä, että yksityinen kopiointi täytyy laillistaa.  Monien
muidenkin laittomien asioiden tehokas valvonta rikkoisi ihmisten
yksityisyyttä, mutta niitä pidetään silti laittomina.  Valvontaa
pitääkin pyrkiä kohdistamaan yksittäisten pienkäyttäjien sijaan
suuriin toimijoihin.  Lataaminen voi toki siirtyä entistä enemmän
suojattuihin anonyymeihin verkkoihin, jolloin rikollisten
jäljittäminen vaikeutuu huomattavasti, mutta samalla tavalla näissä
verkoissa on mahdollista suorittaa muutakin laitonta toimintaa kuten
huumeiden tai varastetun tavaran myyntiä tai sananvapauden rajat
rikkovaa kirjoittelua, uhkailua ja syyttelyä.  Yleisellä tasolla onkin
erittäin mielenkiintoinen ja tärkeä kysymys, voidaanko pitää
laittomana sellaista asiaa, jota ei pystytä valvomaan tai estämään.
% mutta kantoja ei tulisi kohdistaa valikoiden
% tekijänoikeuskysymyksiin.


\section{Muita ehdotuksia kuluttajien oikeuksien parantamiseksi}

Tässä luvussa nostan esille piraattipuolueen ja muiden toimijoiden
esittämiä muita keinoja kuluttajien oikeuksien parantamiseksi ilman,
että tekijöiden suoja ja taloudellinen asema merkittävästi heikkenee.
% Näitä ehdotuksia ovat käsitelleet myös piraattipuolue Ehdotukset ovat
% kirjoittajan eivätkä Piraattipuolueen, mutta näissä kysymyksissä
% näkemykset ovat melko samanlaisia.

\subsection{Tekijänoikeuden suoja-ajan lyhentäminen}

Tekijänoikeuden suoja-ajan olisi hyvä olla mahdollisimman lyhyt, jotta
yhteiskunta voisi rikastua kulttuurin hyödyntämisestä, mutta riittävän
pitkä, jotta laadukkaiden teosten tekijöitä silti löytyisi.  Tällä
hetkellä tekijänoikeuden suoja-aika on 70 vuotta tekijän kuolemasta
(TekL § 43).  Koska merkittävä määrä teoksen tuottamista tuloista
tulee muutaman ensimmäisen vuoden aikana, olisi perusteltua lyhentää
suoja-aikaa esimerkiksi 5--10 vuoteen teoksen
julkaisemisesta.\footnote{Ks. Jokapiraatinoikeus s. 204}


\subsection{Kopiointisuojausten poistaminen}

Tekijöiden mielestä CD/DVD/Bluray-levyä myytäessä ei varsinaisesti
myydä fyysistä levyä vaan sillä olevaa informaatiota, joten nimenomaan
kyseisen informaation eteenpäin jakaminen ja kopioiminen on
kiellettyä.  Kuitenkin voidaan ajatella, että jos kerran myynnin
kohteena on informaatio, tulisi myyjän taata, että kuluttajalla on
mahdollisuus lukea se informaatio ja säilyttää sitä missä muodossa
haluaa.  Sen sijaan että kuluttajilla ei ole \emph{oikeutta} kiertää
kopiointisuojausta, myyjillä pitäisi siis olla \emph{velvollisuus}
antaa myymänsä informaatio ilman kopiointisuojausta tai muita estoja.
%
%kopiosuojauksen poisto vähentää myyntiä
%
%musiikin osalta kopiosuojaukset pääosin poistettu
%
%kopiosuojaus haittaa laillisia käyttäjiä, ei estä piraatteja
%
Käyttöoikeuksien hallintakin voi rajoittaa kuka, missä ja milloin voi
teosta hyödyntää.  Electronier Frontien Finland toteaakin:
``Käyttöestojen hallinnasta tuntuukin vain olevan haittaa maksaville
asiakkaille: musiikin kuuntelu vaikeutuu ja musiikin saa joka
tapauksessa halutessaan
suojaamattomana.''\footnote{Ks. \url{http://www.effi.org/tekijanoikeus/aanitteet/drm.html},
  30.11.2012} Tämän vuoksi estot poistamalla palveltaisiin ennen
kaikkea lainkuuliaisia kuluttajia ja tekijöitä.

%Scientific American


%\subsection{Parempia palveluita}

\subsection{Hyvitysmaksun poistaminen}

Useiden dataa tallentavien laitteiden hintaan sisältyy hyvitysmaksu,
jolla kompensoidaan laillisesta yksityisestä kopioinnista aiheutuvaa
haittaa.  Maksu ei kuitenkaan kohdistu oikein kuluttajiin, sillä
kaikki joutuvat sen maksamaan, vaikka säilyttäisivät hyvitysmaksun
alaisilla laitteilla vain omia kuviaan ja videoitaan.  Maksu ei
kohdistu oikein myöskään tekijöille, sillä rahan jaosta päättää
opetus- ja kulttuuriministeriö, jolla ei kuluttajien yksityisyyden
vuoksi ole tiedossa, keiden teoksia ihmisillä on yksityisesti
kopioituina.  Hyvitysmaksusta olisikin siis parempi luopua ja
mahdolliset välttämättömät tuet voisi ottaa suoraan valtion
budjetista.\footnote{Ks. \url{http://blogi.piraattipuolue.fi/2011/01/11/},
  30.11.2012}


\subsection{Avoimen sisällön tuottaminen}

Tälläkin hetkellä maksulliset teokset kamppailevat laillisesti
ilmaiseksi jaossa olevien teosten
kanssa.\footnote{Ks. Jokapiraatinoikeus s. 91} Voi olla, että
kilpailun kovetessa ainoa keino pärjätä on teosten jakaminen
ilmaiseksi itsekin.  Tarttuvilla avoimilla lisensseillä
(esim. CC-BY-SA ja GPL) on mahdollista luoda teoksia, joita saa
vapaasti käyttää ja jakaa kunhan sallii sen muillekin.  Tämä avoin
sisältömaailma on erillinen suljettujen teosten maailmasta, joten
suljettujen teosten maailma voi ajan kuluessa näivettyä, jos avointen
lisenssien maailmaan tulee riittävän kattava ja laadukas valikoima.
Avoimien teoksien maailma toimii nykyisten tekijänoikeuspykälien
varassa, joten mitään lakimuutoksia tämän osalta ei edes tarvita --
ainoastaan tekijöiden halu tuottaa avointa sisältöä.

% cc-by-sa gpl

% saadaan luotua nykyisiä pykäliä hyödyntämällä oma vapaa maailma

% Mikäli laillisesti on ilmaiseksi saatavilla riittävä valikoima
% riittävän laadukasta sisältöä, maksullinen sisältö joutuu kriisiin.

% julkinen data (esim. karttapalvelut)

\section{Yhteenveto}

Tässä raportissa on tarkasteltu piraattipuolueen ehdotusta kaiken
yksityisen kopioinnin laillistamiseksi.  Piraattipuolueen radikaalin
ehdotuksen perustelut vaikuttivat kuitenkin riittämättömiltä
verrattuna mahdollisiin haittoihin, minkä takia raportissa esiteltiin
muita ehdotuksia, joilla kuluttajien asemaa voitaisiin parantaa ilman,
että tekijänoikeuden suojaa tarvitsee yksityisen käytön osalta
kokonaan poistaa.  Näillä ehdotuksilla -- esimerkiksi suoja-ajan
lyhentämisellä -- voisi olla suuri merkitys yhteiskunnan
kokonaishyödyn kasvattamisessa.

%Yksityisen kopioinnin salliminen ongelmallista

%Muilla korjaustoimenpiteillä saavutetaan jo merkittäviä hyötyjä.

% Lähdeluettelo
\pagebreak
\bibliography{viitteet}
% \section*{Lähteet}

% \begin{thebibliography}
  
% \bibitem{lehdisto1}
%   huh%\url{http://piraattipuolue.fi/2011/10/piraattipuolue-tuomitsee-raesaesen-puheet-nettisensuurin-laajentamisesta}.

% \bibitem{lehdisto2}
%   hah%\url{http://piraattipuolue.fi/2011/10/the-pirate-bay-sensuuri-uhka-sananvapaudelle}.
    
% \end{thebibliography}

\end{document}

%%% Local Variables: 
%%% mode: latex
%%% TeX-master: t
%%% End: 
